\documentclass[a4 paper]{article}
\usepackage[inner=2.0cm,outer=2.0cm,top=2.5cm,bottom=2.5cm]{geometry}
\usepackage{setspace}
\usepackage[ruled]{algorithm2e}
\usepackage[rgb]{xcolor}
\usepackage{verbatim}
\usepackage{subcaption}
\usepackage{amsgen,amsmath,amstext,amsbsy,amsopn,tikz,amssymb,tkz-linknodes}
\usepackage{fancyhdr}
\usepackage[colorlinks=true, urlcolor=blue,  linkcolor=blue, citecolor=blue]{hyperref}
\usepackage[colorinlistoftodos]{todonotes}
\usepackage{rotating}
\usepackage{booktabs}
\newcommand{\ra}[1]{\renewcommand{\arraystretch}{#1}}

\newtheorem{thm}{Theorem}[section]
\newtheorem{prop}[thm]{Proposition}
\newtheorem{lem}[thm]{Lemma}
\newtheorem{cor}[thm]{Corollary}
\newtheorem{defn}[thm]{Definition}
\newtheorem{rem}[thm]{Remark}
\numberwithin{equation}{section}

\newcommand{\homework}[6]{
   \pagestyle{myheadings}
   \thispagestyle{plain}
   \newpage
   \setcounter{page}{1}
   \noindent
   \begin{center}
   \framebox{
      \vbox{\vspace{2mm}
    \hbox to 6.28in { {\bf CSE 211:~Discrete Mathematics \hfill {\small (#2)}} }
       \vspace{6mm}
       \hbox to 6.28in { {\Large \hfill #1  \hfill} }
       \vspace{6mm}
       \hbox to 6.28in { {\it Instructor: {\rm #3} \hfill Name: {\rm #5} \hfill Student Id: {\rm #6}} \hfill}
       \hbox to 6.28in { {\it Assistant: #4  \hfill #6}}
      \vspace{2mm}}
   }
   \end{center}
   \markboth{#5 -- #1}{#5 -- #1}
   \vspace*{4mm}
}

\newcommand{\problem}[2]{~\\\fbox{\textbf{Problem #1}}\hfill (#2 points)\newline\newline}
\newcommand{\subproblem}[1]{~\newline\textbf{(#1)}}
\newcommand{\D}{\mathcal{D}}
\newcommand{\Hy}{\mathcal{H}}
\newcommand{\VS}{\textrm{VS}}
\newcommand{\solution}{~\newline\textbf{\textit{(Solution)}} }

\newcommand{\bbF}{\mathbb{F}}
\newcommand{\bbX}{\mathbb{X}}
\newcommand{\bI}{\mathbf{I}}
\newcommand{\bX}{\mathbf{X}}
\newcommand{\bY}{\mathbf{Y}}
\newcommand{\bepsilon}{\boldsymbol{\epsilon}}
\newcommand{\balpha}{\boldsymbol{\alpha}}
\newcommand{\bbeta}{\boldsymbol{\beta}}
\newcommand{\0}{\mathbf{0}}


\begin{document}
\homework{Homework \#4}{Due: 24/12/19}{Dr. Zafeirakis Zafeirakopoulos}{Gizem S\"ung\"u, Başak Karakaş}{}{}
\textbf{Course Policy}: Read all the instructions below carefully before you start working on the assignment, and before you make a submission.
\begin{itemize}
\item It is not a group homework. Do not share your answers to anyone in any circumstance. Any cheating means at least -100 for both sides. 
\item Do not take any information from Internet.
\item No late homework will be accepted. 
\item For any questions about the homework, send an email to gizemsungu@gtu.edu.tr
\item Submit your homework into Assignments/Homework4 directory of the CoCalc project CSE211-2019-2020.
\end{itemize}

\problem{1: Nonhomogeneous Linear Recurrence Relations}{15+15=30}
Consider the nonhomogeneous linear recurrence relation $a_n$ = 3$a_{n-1}$ + $2^n$ .\\
\subproblem{a} Show that whether $a_n$ = $-2^{n+1}$ is a solution of the given recurrence relation or not. Show your work step by step.\\
\solution
\newline
1)\newline
if  $a_n$ = $-2^{n+1}$ \newline
$a_{n-1}$ = $-2^{n}$ \newline
2) \newline
$-2^{n+1}$ = -3.$2^{n}$ + $2^n$ \newline
$-2^{n+1}$ = $-2.2^{n}$ \newline
3)$-2^{n+1}$ =$-2^{n+1}$  Yes $a_n$ = $-2^{n+1}$ is a solution of the given recurrence relation. \newline

\subproblem{b} Find the solution with $a_0$ = 1.\\
\solution
\newline
1) $a_n$ = $a_n$ $(h)$ + $a_n$ $(p)$\newline
2)For  $a_n$ $(h)$ \newline
$a_n$ = 3$a_{n-1}$ \newline
if we assume that  $a_{n-1}$ = 1  and $a_n$ =r \newline
r = 3.1  $=>$ root \newline
$a_n$ = $c_1$.$3^{n}$ + $a_n$ $(p)$ \newline
3) For $a_n$ $(p)$ \newline
if we assume that $a_n$ = A.$2^{n}$ \newline
$a_{n-1}$ = A.$2^{n-1}$ \newline
if we put this equation to general relation we get;\newline
A.$2^{n}$ = 3.A.$2^{n-1}$ + $2^{n}$ \newline
A = 3A/2 +1 \newline
A = -2 \newline
$a_n$ $(p)$ = -2.$2^{n-1}$ \newline
4) $a_n$ $(general)$ = $c_1$.$3^{n}$ - 2.$2^{n-1}$ \newline
For n= 0 $a_0$ = $c_1$ - 2 = 1 \newline
$c_1$ = 3 \newline
5)  $a_n$ $(general)$ = 3.$3^{n}$ - 2.$2^{n-1}$ =$3^{n}$ - 2.$2^{n-1}$  \newline


\problem{2: Linear Recurrence Relations}{35}
Find all solutions of the recurrence relation $a_n$ = 7$a_{n-1}$ - 16$a_{n-2}$ + 12$a_{n-3}$ + n$4^n$ with $a_0$ = -2, $a_1$ = 0, and $a_2$ = 5.\\
\solution
\newline
1)This relation is nonhomogenous relation so relation have homogenous and partial part.\newline
$a_n$ $(g)$ = $a_n$ $(h)$ + $a_n$ $(p)$ \newline
1)$a_n$ $(h)$ =  7$a_{n-1}$ - 16$a_{n-2}$ + 12$a_{n-3}$ \newline


\problem{3: Linear Homogeneous Recurrence Relations }{20+15 = 35}
Consider the linear homogeneous recurrence relation $a_n$ = 2$a_{n-1}$ - 2$a_{n-2}$.
\subproblem{a} Find the characteristic roots of the recurrence relation.\\
\solution
\newline
1)if we assume that $a_{n-2}$ = 1 , $a_{n-1}$ =r  and $a_n$=$r^2$ \newline
$r^2$ -2r +2 =0 \newline
$\Delta$ = $b^2$ - 4.a.c\newline
$\Delta$ = -4 so  roots of the relation is complex numbers \newline
r = -$(-2)$ +$\sqrt{\triangle}$ / 2.a \newline
r = -$(-2)$ +$\sqrt{\triangle}$ / 2.a \newline
$r_1$ =-$(-2)$ +$\sqrt{-4}$ / 2 = 1+i \newline
$r_2$ =-$(-2)$ -$\sqrt{-4}$ / 2 = 1-i \newline

\subproblem{b} Find the solution of the recurrence relation with $a_0$ = 1 and $a_1$ = 2.\\
\solution
\newline
$a_{n}$ $(h)$ = $c_1$ . $(1+i)$ $^n$ + $c_2$ . $(1-i)$ $^n$  \newline
for n= 0 \newline
1 =  $c_1$ +  $c_2$ \newline
for n=1 \newline
2 = $c_1$ . $(1+i)$ +  $c_2$ . $(1-i)$ \newline
$c_1$ + $c_2$ + i.($c_1$ - $c_2$ ) = 2 \newline
$c_1$ - $c_2$ = -i \newline
$c_1$ + $c_2$ = 1 \newline \newline
$c_1$ = $\frac{1-i}{2}$ and  $c_2$=$\frac{1+i}{2}$ so \newline \newline
$a_{n}$ $(h)$ = $\frac{1-i}{2}$ . $(1+i)$ $^n$ + $\frac{1+i}{2}$ . $(1-i)$ $^n$  \newline


\end{document} 


\documentclass[a4 paper]{article}
\usepackage[inner=2.0cm,outer=2.0cm,top=2.5cm,bottom=2.5cm]{geometry}
\usepackage{setspace}
\usepackage[ruled]{algorithm2e}
\usepackage[rgb]{xcolor}
\usepackage{verbatim}
\usepackage{subcaption}
\usepackage{amsgen,amsmath,amstext,amsbsy,amsopn,tikz,amssymb,tkz-linknodes}
\usepackage{fancyhdr}
\usepackage[colorlinks=true, urlcolor=blue,  linkcolor=blue, citecolor=blue]{hyperref}
\usepackage[colorinlistoftodos]{todonotes}
\usepackage{rotating}
\usepackage{booktabs}
\newcommand{\ra}[1]{\renewcommand{\arraystretch}{#1}}

\newtheorem{thm}{Theorem}[section]
\newtheorem{prop}[thm]{Proposition}
\newtheorem{lem}[thm]{Lemma}
\newtheorem{cor}[thm]{Corollary}
\newtheorem{defn}[thm]{Definition}
\newtheorem{rem}[thm]{Remark}
\numberwithin{equation}{section}

\newcommand{\homework}[6]{
   \pagestyle{myheadings}
   \thispagestyle{plain}
   \newpage
   \setcounter{page}{1}
   \noindent
   \begin{center}
   \framebox{
      \vbox{\vspace{2mm}
    \hbox to 6.28in { {\bf CSE 211:~Discrete Mathematics \hfill {\small (#2)}} }
       \vspace{6mm}
       \hbox to 6.28in { {\Large \hfill #1  \hfill} }
       \vspace{6mm}
       \hbox to 6.28in { {\it Instructor: {\rm #3} \hfill Name: {\rm #5} \hfill Student Id: {\rm #6}} \hfill}
       \hbox to 6.28in { {\it Assistant: #4  \hfill #6}}
      \vspace{2mm}}
   }
   \end{center}
   \markboth{#5 -- #1}{#5 -- #1}
   \vspace*{4mm}
}

\newcommand{\problem}[2]{~\\\fbox{\textbf{Problem #1}}\hfill (#2 points)\newline\newline}
\newcommand{\subproblem}[1]{~\newline\textbf{(#1)}}
\newcommand{\D}{\mathcal{D}}
\newcommand{\Hy}{\mathcal{H}}
\newcommand{\VS}{\textrm{VS}}
\newcommand{\solution}{~\newline\textbf{\textit{(Solution)}} }

\newcommand{\bbF}{\mathbb{F}}
\newcommand{\bbX}{\mathbb{X}}
\newcommand{\bI}{\mathbf{I}}
\newcommand{\bX}{\mathbf{X}}
\newcommand{\bY}{\mathbf{Y}}
\newcommand{\bepsilon}{\boldsymbol{\epsilon}}
\newcommand{\balpha}{\boldsymbol{\alpha}}
\newcommand{\bbeta}{\boldsymbol{\beta}}
\newcommand{\0}{\mathbf{0}}


\begin{document}
\homework{Homework \#4}{Due: 24/12/19}{Dr. Zafeirakis Zafeirakopoulos}{Gizem S\"ung\"u, Başak Karakaş}{}{}
\textbf{Course Policy}: Read all the instructions below carefully before you start working on the assignment, and before you make a submission.
\begin{itemize}
\item It is not a group homework. Do not share your answers to anyone in any circumstance. Any cheating means at least -100 for both sides. 
\item Do not take any information from Internet.
\item No late homework will be accepted. 
\item For any questions about the homework, send an email to gizemsungu@gtu.edu.tr
\item Submit your homework into Assignments/Homework4 directory of the CoCalc project CSE211-2019-2020.
\end{itemize}

\problem{1: Nonhomogeneous Linear Recurrence Relations}{15+15=30}
Consider the nonhomogeneous linear recurrence relation $a_n$ = 3$a_{n-1}$ + $2^n$ .\\
\subproblem{a} Show that whether $a_n$ = $-2^{n+1}$ is a solution of the given recurrence relation or not. Show your work step by step.\\
\solution
\newline
1)\newline
if  $a_n$ = $-2^{n+1}$ \newline
$a_{n-1}$ = $-2^{n}$ \newline
2) \newline
$-2^{n+1}$ = -3.$2^{n}$ + $2^n$ \newline
$-2^{n+1}$ = $-2.2^{n}$ \newline
3)$-2^{n+1}$ =$-2^{n+1}$  Yes $a_n$ = $-2^{n+1}$ is a solution of the given recurrence relation. \newline

\subproblem{b} Find the solution with $a_0$ = 1.\\
\solution
\newline
1) $a_n$ = $a_n$ $(h)$ + $a_n$ $(p)$\newline
2)For  $a_n$ $(h)$ \newline
$a_n$ = 3$a_{n-1}$ \newline
if we assume that  $a_{n-1}$ = 1  and $a_n$ =r \newline
r = 3.1  $=>$ root \newline
$a_n$ = $c_1$.$3^{n}$ + $a_n$ $(p)$ \newline
3) For $a_n$ $(p)$ \newline
if we assume that $a_n$ = A.$2^{n}$ \newline
$a_{n-1}$ = A.$2^{n-1}$ \newline
if we put this equation to general relation we get;\newline
A.$2^{n}$ = 3.A.$2^{n-1}$ + $2^{n}$ \newline
A = 3A/2 +1 \newline
A = -2 \newline
$a_n$ $(p)$ = -2.$2^{n-1}$ \newline
4) $a_n$ $(general)$ = $c_1$.$3^{n}$ - 2.$2^{n-1}$ \newline
For n= 0 $a_0$ = $c_1$ - 2 = 1 \newline
$c_1$ = 3 \newline
5)  $a_n$ $(general)$ = 3.$3^{n}$ - 2.$2^{n-1}$ =$3^{n}$ - 2.$2^{n-1}$  \newline


\problem{2: Linear Recurrence Relations}{35}
Find all solutions of the recurrence relation $a_n$ = 7$a_{n-1}$ - 16$a_{n-2}$ + 12$a_{n-3}$ + n$4^n$ with $a_0$ = -2, $a_1$ = 0, and $a_2$ = 5.\\
\solution
\newline
\newline


\problem{3: Linear Homogeneous Recurrence Relations }{20+15 = 35}
Consider the linear homogeneous recurrence relation $a_n$ = 2$a_{n-1}$ - 2$a_{n-2}$.
\subproblem{a} Find the characteristic roots of the recurrence relation.\\
\solution
\newline
1)if we assume that $a_{n-2}$ = 1 , $a_{n-1}$ =r  and $a_n$=$r^2$ \newline
$r^2$ = 
\subproblem{b} Find the solution of the recurrence relation with $a_0$ = 1 and $a_1$ = 2.\\
\solution


\end{document} 


