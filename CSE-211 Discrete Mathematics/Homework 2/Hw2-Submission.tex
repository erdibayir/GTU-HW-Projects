\documentclass[a4 paper]{article}
\usepackage[inner=2.0cm,outer=2.0cm,top=2.5cm,bottom=2.5cm]{geometry}
\usepackage{setspace}
\usepackage[ruled]{algorithm2e}
\usepackage[rgb]{xcolor}
\usepackage{verbatim}
\usepackage{subcaption}
\usepackage{amsgen,amsmath,amstext,amsbsy,amsopn,tikz,amssymb,tkz-linknodes}
\usepackage{fancyhdr}
\usepackage[colorlinks=true, urlcolor=blue,  linkcolor=blue, citecolor=blue]{hyperref}
\usepackage[colorinlistoftodos]{todonotes}
\usepackage{rotating}
\usepackage{booktabs}
\newcommand{\ra}[1]{\renewcommand{\arraystretch}{#1}}

\newtheorem{thm}{Theorem}[section]
\newtheorem{prop}[thm]{Proposition}
\newtheorem{lem}[thm]{Lemma}
\newtheorem{cor}[thm]{Corollary}
\newtheorem{defn}[thm]{Definition}
\newtheorem{rem}[thm]{Remark}
\numberwithin{equation}{section}

\newcommand{\homework}[6]{
   \pagestyle{myheadings}
   \thispagestyle{plain}
   \newpage
   \setcounter{page}{1}
   \noindent
   \begin{center}
   \framebox{
      \vbox{\vspace{2mm}
    \hbox to 6.28in { {\bf CSE 211:~Discrete Mathematics \hfill {\small (#2)}} }
       \vspace{6mm}
       \hbox to 6.28in { {\Large \hfill #1  \hfill} }
       \vspace{6mm}
       \hbox to 6.28in { {\it Instructor: {\rm #3} \hfill Name: {\rm #5} \hfill Student Id: {\rm #6}} \hfill}
       \hbox to 6.28in { {\it Assistant: #4  \hfill #6}}
      \vspace{2mm}}
   }
   \end{center}
   \markboth{#5 -- #1}{#5 -- #1}
   \vspace*{4mm}
}

\newcommand{\problem}[2]{~\\\fbox{\textbf{Problem #1}}\hfill (#2 points)\newline\newline}
\newcommand{\subproblem}[1]{~\newline\textbf{(#1)}}
\newcommand{\D}{\mathcal{D}}
\newcommand{\Hy}{\mathcal{H}}
\newcommand{\VS}{\textrm{VS}}
\newcommand{\solution}{~\newline\textbf{\textit{(Solution)}} }

\newcommand{\bbF}{\mathbb{F}}
\newcommand{\bbX}{\mathbb{X}}
\newcommand{\bI}{\mathbf{I}}
\newcommand{\bX}{\mathbf{X}}
\newcommand{\bY}{\mathbf{Y}}
\newcommand{\bepsilon}{\boldsymbol{\epsilon}}
\newcommand{\balpha}{\boldsymbol{\alpha}}
\newcommand{\bbeta}{\boldsymbol{\beta}}
\newcommand{\0}{\mathbf{0}}


\begin{document}
\homework{Homework \#2}{Due: 12/11/19}{Dr. Zafeirakis Zafeirakopoulos}{Gizem S\"ung\"u}{}{}
\textbf{Course Policy}: Read all the instructions below carefully before you start working on the assignment, and before you make a submission.
\begin{itemize}
\item It is not a group homework. Do not share your answers to anyone in any circumstance. Any cheating means at least -100 for both sides. 
\item Do not take any information from Internet.
\item No late homework will be accepted. 
\item For any questions about the homework, send an email to gizemsungu@gtu.edu.tr
\item Submit your homework into Assignments/Homework1 directory of the CoCalc project CSE211-2019-2020.
\end{itemize}

\problem{1: Sets}{2+2+2+2+2=10}
Which of the following sets are equal? Show your work step by step.\newline
\subproblem{a} $\{$t : t is a root of $x^2$ – 6x + 8 = 0$\}$
\newline

\subproblem{b} $\{$y : y is a real number in the closed interval [2, 3]$\}$
\newline
\subproblem{c} $\{$4, 2, 5, 4$\}$
\newline
\subproblem{d} $\{$4, 5, 7, 2$\}$ - $\{$5, 7$\}$
\newline

\subproblem{e} $\{$q: q is either the number of sides of a rectangle or the number of digits in any integer between 11 and 99$\}$
\solution
\newline
a) t= 4 or t= 2 S1= \{4,2\} \newline
b) \{2.00 , 2.01 , 2.02 ...\} \newline
d) \{4,2\} \newline
e)Sides of rectangle=4  Number of Digits=2 \newline
\{4,2\} \newline
a , d and e  are equal \newline

\newpage
\problem{2: Cartesian Product of Sets}{15}
Explain why (A $\times$ B) $\times$ (C $\times$ D) and A $\times$ (B $\times$ C) $\times$ D are not the same.
\solution
\newline
A = \{A1\}, B = \{B1\} , C = \{C1\} , D = \{D1\} \newline \newline
A $\times$ B = \{(A1 ,B1)\}   (C $\times$ D) = \{(C1 ,D1)\} \newline
(A $\times$ B) $\times$ (C $\times$ D) = \{(A1 ,B1) , (C1 ,D1)\} \newline
(B $\times$ C) = \{(B1 ,C1)\}   \newline    A $\times$ (B $\times$ C)= \{(A1 ,B1) ,(A1,C1)\} \newline
A $\times$ (B $\times$ C) $\times$ D = \{(A1 ,D1) ,(B1,D1) , (C1,D1)\} \newline
They are not same \newline
\newline
\problem{3: Cartesian Product of Sets in Algorithms }{25}
Let A, B and C be sets which have different cardinalities. Let (p, q, r) be each triple of A $\times$ B $\times$ C where p $\in$ A, q $\in$ B and r $\in$ C. Design an algorithm which finds all the triples that are satisfying the criteria: p $\leq$ q and q $\geq$ r. Write the pseudo code of the algorithm in your solution.\newline
\newline
For example: Let the set A, B and C be as A = $\{$ 3, 5, 7 $\}$, B = $\{$ 3, 6 $\}$ and C = $\{$ 4, 6, 9 $\}$. Then the output should be : $\{$ (3, 6, 4), (3, 6, 6), (5, 6, 4), (5, 6, 6) $\}$. \newline
\newline
(Note: Assume that you have sets of A, B, C as an input argument.)\newline
\solution

\begin{algorithm}
\SetAlgoLined
\KwIn{The sets of A, B, C}
\eIf{write a condition}{
    Statements
}{
 Statements
}
 When you want to write a for loop, you can use: \newline
\For{write a condition}{

}
 When you want to write a while loop, you can use: \newline
\While{write a condition}{
If you need to return, use \Return
}
 For any additional things you have to do while writing your pseudo code, Google "How to use algorithm2e in Latex?".
\caption{Pseudo Code of Your Algorithm}
\end{algorithm}


\newpage
\problem{4: Relations}{3+3+3+3+3+3+3=21}
Determine whether the relation R on the set of all integers is reflexive, symmetric, antisymmetric, and/or transitive, where (x, y) $\in$ R if and only if
\subproblem{a} x $\neq$ y.
\solution
\newline
\textbf{Reflexive}\newline
We assume that(x,y) $\rightarrow$ (a,a) for reflexive a $\neq$ a is false\newline
It isn't reflexive\newline
\textbf{Symetric}\newline
x $\neq$ y if we assume that (x,y)  $\rightarrow$  (y,x) y $\neq$ x is true so it is symetric\newline
\textbf{Antisymetric}\newline
(x, y) $\in$ R and x  $\neq$ y ex:(2,3) and (3,2) are elements of this relation.It is symetric and It isn't antisymetric.\newline
\textbf{Transitive} \newline
for (a,b)  $\rightarrow$  a $\neq$ b for (b,c)  $\rightarrow$  b $\neq$ c for (a,c) $\rightarrow$  a $\neq$ c\newline
There are elements that provide this condition, but not all. For example:
(1,3) and (3,1) are in this relation but (1,1) isn't  in relation.This relation isn't transitive
\newline
\subproblem{b} xy $\geq$ 1.
\solution
\newline
\textbf{Reflexive}\newline
if we assume that (x,y) $\rightarrow$ (a,a)== a.a $\geq$ 1\newline
(0,0) doesn't provide this condition.It isn't reflexive\newline
\textbf{Symetric}\newline
for (a,b)  $\rightarrow$  a.b $\geq$ 1 for (b,a)  $\rightarrow$  b.a $\geq$ 1.Both are  same so this relation is symetric.\newline
\textbf{Antisymetric}\newline
Ex:(2,3) and (3,2) are elements of this relation.It is symetric so This relatino isn't antisymetric.\newline
\textbf{Transitive}\newline
for (a,b)  $\rightarrow$  a.b $\geq$ 1 (b,c) $\rightarrow$  b.c $\geq$ 1 for (a,c) $\rightarrow$  a.c $\geq$ 1 is true.This relation is transitive\newline
\newline
\subproblem{c} x = y + 1 or x = y - 1.
\solution
\newline
\textbf{Reflexive}\newline
we assume that (x,y) $\rightarrow$  (a,a)\newline
a=a+1 or a=a-1  F or F = F.This relation isn't reflexive\newline
\textbf{Symetric}\newline
Check (a,b) and (b,a)\newline
(a,b)\newline
a=b+1 or a= b-1 If organized b=a-1 or b=a+1\newline
(b,a)\newline
b=a+1 or b=a-1\newline
(a,b) and (b,a) are same and true.This relation symetric\newline
\textbf{Antisymetric}\newline
Ex:(2,1) in this relation  and (1,2) are also in this relation\newline
This is not Antisymetric as we have proved in Symetric.\newline
\textbf{Transitive}\newline
(2,1) and (1,2) are elements of relation but (2,2) isn't element of relation.This relation isn't transitive.
\newline
\subproblem{d} x is a multiple of y.
\solution
\newline
\textbf{Reflexive}\newline
x=k.y , k$\in$ Z\newline
(a,a)  $\rightarrow$ a=k.a  k$\in$ Z   k=1.It is true.Relation is reflexive.\newline
\textbf{Symetric}\newline
For (a,b)  $\rightarrow$  a=k.b , k $\in$ Z    $(a\div b)$=k\newline
For (b,a)  $\rightarrow$  b=m.a ,m $\in$ Z    $(a\div b)$=$(1\div m)$ \newline
$(1\div m)$ and k must be $\in$ Z but 1/m not $\in$ Z.Relation is not symetric.\newline
\newline
\textbf{Antisymetric}\newline
Relation doesn't have any symetric element except (1,1) and (1,1) also provide Antisymetric.Relation is Antisymetric. \newline
\textbf{Transitive}\newline
(a,b) $\rightarrow$ a=k.b (b,c) $\rightarrow$ b=m.c  (a,c) $\rightarrow$ a=n.c k,m,n $\in$ Z if a is multiple of b also a is multiple of c\newline Relation is transitive.\newline
\subproblem{e} x and y are both negative or both nonnegative.
\solution
\newline
x $<$ 0 and y $<$ 0 OR  x $>$ 0 and y $>$ 0\newline
\textbf{Reflexive}\newline
(a,a) $\rightarrow$ a $<$ 0 and a $<$ 0 OR  a $>$ 0 and a $>$ 0.It is true relation is reflexive\newline 
\textbf{Symetric}\newline
(a,b) $\rightarrow$ a $<$ 0 and b $<$ 0 OR  a $>$ 0 and b $>$ 0\newline
(b,a) $\rightarrow$ b $<$ 0 and a $<$ 0 OR  b $>$ 0 and a $>$ 0\newline
They are same so relation is symetric\newline
\textbf{Antisymetric}\newline
Relaiton have symetric elemets like:(1,2) and (2,1) both of them already in relation.So relation isn't antisymetric\newline
\textbf{Transitive}\newline
(a,b) $\rightarrow$ a $<$ 0 and b $<$ 0 OR  a $>$ 0 and b $>$ 0\newline
(b,c) $\rightarrow$ b $<$ 0 and c $<$ 0 OR  b $>$ 0 and c $>$ 0\newline
(a,c) $\rightarrow$ a $<$ 0 and c $<$ 0 OR  a $>$ 0 and c $>$ 0\newline
If a $<$ 0 b must be $<$ 0 then c $<$ 0 So (a,c) is have to in relation.\newline
So relation is transitive.\newline
\newline
\subproblem{f} x $\geq$ $y^2$.
\solution
\newline
\textbf{Reflexive}\newline
(a,a) $\rightarrow$ a $\geq$ $a^2$ It is not true for negative numbers.Relation isn't reflexive\newline
\textbf{Symetric}\newline
(a,b) $\rightarrow$ a $\geq$ $b^2$ \newline
(b,a) $\rightarrow$ b $\geq$ $a^2$ This situation is false.Relation isn't symetric\newline 
\textbf{Antisymetric}\newline
Relation only have (1,1) (2,2) (3,3)... symetric elements and this does not disturb the antisymmetric condition.Relation is antisymetric.\newline
\newline
\textbf{Transitive}\newline
(a,b) $\rightarrow$ a $\geq$ $b^2$ \newline
(b,c) $\rightarrow$ b $\geq$ $c^2$ \newline
(a,c) $\rightarrow$ a $\geq$ $c^2$ \newline
a $>$ $b^2$ also a $c^2$.(a,c) in relation elements,Relation is transitive.
\subproblem{g} x = $y^2$.
\solution
\newline
\textbf{Reflexive}\newline
(x,y) $\rightarrow$ (a,a) a = $a^2$. \newline
It only provide  (0,0) and (1,1) other elements like (2,2) isn't in relation .Relation isn't reflexive\newline
\textbf{Symetric}\newline
(a,b) $\rightarrow$  a = $b^2$ \newline
(b,a) $\rightarrow$  b = $a^2$ \newline
It only true for (1,1) other elements like (4,2) is in relation but (2,4) isn't in relation.Relation isn't symetric.
\textbf{Antisymetric}\newline
Relation antisymetric.Because relation have only (1,1) symetric elemets and this does not disturb the antisymmetric condition.\newline
\newline
%%%%%%REMOVE \newline commands while writing your answer%%%%%
\newline
\newline


\problem{5: Functions}{15}
If f and f $\circ$ g are one-to-one, does it follow that g is one-to-one? Justify your answer.
\solution
\newline
\newline
We suppose x1 =x2 \newline
f o g(x1) = f  o g (x2) \newline
f(g(x1)) = f(g(x2)) then g(x1) =g(x2)\newline
g is one-to-one\newline
\newline
\problem{6: Inverse of Functions}{7+7=14}
Let f be the function from $\mathbb{R}$ to $\mathbb{R}$ defined by f(x) = $x^2$. Find
\subproblem{a} $f^{-1}$ ($\{$ x $\vert$ 0 $<$ x $<$ 1 $\}$)
\solution
\newline
\newline
y=$x^2$ $\rightarrow$ x=$\sqrt{y}$  $\rightarrow$ $f^{-1}$(x) = $\sqrt{x}$ 
\newline
\subproblem{b}$f^{-1}$ ($\{$ x $\vert$ x $>$ 4 $\}$) 
\solution
\newline
for x: x $>$ 4 also
$f^{-1}$(x) = $\sqrt{x}$ \newline
\end{document}