\documentclass[a4 paper]{article}
\usepackage[inner=2.0cm,outer=2.0cm,top=2.5cm,bottom=2.5cm]{geometry}
\usepackage{setspace}
\usepackage[rgb]{xcolor}
\usepackage{verbatim}
\usepackage{subcaption}
\usepackage{amsgen,amsmath,amstext,amsbsy,amsopn,tikz,amssymb,tkz-linknodes}
\usepackage{fancyhdr}
\usepackage[colorlinks=true, urlcolor=blue,  linkcolor=blue, citecolor=blue]{hyperref}
\usepackage[colorinlistoftodos]{todonotes}
\usepackage{rotating}
\usepackage{booktabs}
\newcommand{\ra}[1]{\renewcommand{\arraystretch}{#1}}

\newtheorem{thm}{Theorem}[section]
\newtheorem{prop}[thm]{Proposition}
\newtheorem{lem}[thm]{Lemma}
\newtheorem{cor}[thm]{Corollary}
\newtheorem{defn}[thm]{Definition}
\newtheorem{rem}[thm]{Remark}
\numberwithin{equation}{section}

\newcommand{\homework}[6]{
   \pagestyle{myheadings}
   \thispagestyle{plain}
   \newpage
   \setcounter{page}{1}
   \noindent
   \begin{center}
   \framebox{
      \vbox{\vspace{2mm}
    \hbox to 6.28in { {\bf CSE 211:~Discrete Mathematics \hfill {\small (#2)}} }
       \vspace{6mm}
       \hbox to 6.28in { {\Large \hfill #1  \hfill} }
       \vspace{6mm}
       \hbox to 6.28in { {\it Instructor: {\rm #3} \hfill Name:  {\rm #5} \hfill Student Id: {\rm #6}} \hfill}
       \hbox to 6.28in { {\it Assistant: #4  \hfill #6}}
      \vspace{2mm}}
   }
   \end{center}
   \markboth{#5 -- #1}{#5 -- #1}
   \vspace*{4mm}
}

\newcommand{\problem}[2]{~\\\fbox{\textbf{Problem #1}}\hfill (#2 points)\newline\newline}
\newcommand{\subproblem}[1]{~\newline\textbf{(#1)}}
\newcommand{\D}{\mathcal{D}}
\newcommand{\Hy}{\mathcal{H}}
\newcommand{\VS}{\textrm{VS}}
\newcommand{\solution}{~\newline\textbf{\textit{(Solution)}} }

\newcommand{\bbF}{\mathbb{F}}
\newcommand{\bbX}{\mathbb{X}}
\newcommand{\bI}{\mathbf{I}}
\newcommand{\bX}{\mathbf{X}}
\newcommand{\bY}{\mathbf{Y}}
\newcommand{\bepsilon}{\boldsymbol{\epsilon}}
\newcommand{\balpha}{\boldsymbol{\alpha}}
\newcommand{\bbeta}{\boldsymbol{\beta}}
\newcommand{\0}{\mathbf{0}}


\begin{document}
\homework{Homework \#1}{Due: 27/10/19}{Dr. Zafeirakis Zafeirakopoulos}{Gizem S\"ung\"u}{}{}
\textbf{Course Policy}: Read all the instructions below carefully before you start working on the assignment, and before you make a submission.
\begin{itemize}
\item It is not a group homework. Do not share your answers to anyone in any circumstance. Any cheating means at least -100 for both sides. 
\item Do not take any information from Internet.
\item No late homework will be accepted. 
\item For any questions about the homework, send an email to gizemsungu@gtu.edu.tr
\item Submit your homework into Assignments/Homework1 directory of the CoCalc project CSE211-2019-2020.
\end{itemize}

\problem{1: Conditional Statements}{5+5+5=15}
State the converse, contrapositive, and inverse of each of these conditional statements.


\subproblem{a} If it snows tonight, then I will stay at home.
\solution
%%%%%%REMOVE \newline commands while writing your answer%%%%%
\newline
\newline
\textbf{Converse: If I stay at home , it will snow tonight.}
\newline
\newline
\textbf{Contrapositive: If I don’t stay at home , it won’t snow tonight.}
\newline
\newline
\textbf{Inverse:If it doesn’t snow tonight, I won’t stay at home. }
\newline

\subproblem{b} I go to the beach whenever it is a sunny summer day.
\solution
%%%%%%REMOVE \newline commands while writing your answer%%%%%
\newline
\newline
\textbf{Converse:It is a sunny summer day whenever I go to the beach.}
\newline
\newline
\textbf{Contrapositive:It is not a sunny summer day whenever I  don’t go to the beach. }
\newline
\newline
\textbf{Inverse:I don’t go to the beach whenever it is not a sunny summer day. }
\newline



\subproblem{c} When I stay up late, it is necessary that I sleep until
noon.
\solution
%%%%%%REMOVE \newline commands while writing your answer%%%%%
\newline
\newline
\textbf{Converse: When It is necessary that I sleep until noon ,I stay up late.}
\newline
\newline
\textbf{Contrapositive:When It is not necessary that I sleep until noon,I don’t stay up late. }
\newline
\newline
\textbf{Inverse:When I don’t stay up late , it is not necessary that I sleep until noon. }
\newline
\newline

\newpage
\problem{2: Truth Tables For Logic Operators}{5+5+5=15}
Construct a truth table for each of the following compound propositions.
\subproblem{a} (p $\oplus$ $\neg$ q)
\solution
%%%%%%REMOVE \newline commands while writing your answer%%%%%
\newline
\begin{displaymath}[]
\begin{array}{|l|l|l|l|}
\hline
\multicolumn{1}{|c|}{p} & q & $\neg$ q & (p  $\oplus$ $\neg$ q) \\ \hline
T & T & F & T \\ \hline
T & F & T & F \\ \hline
F & T & F & F \\ \hline
F & F & T & T \\ \hline
\end{array}
\end{displaymath}
\subproblem{b} (p $\iff$ q) $\oplus$ ( $\neg$ p $\iff$ $\neg$ r)
\solution
\newline
\begin{displaymath}[]
\begin{array}{|l|l|l|l|l|l|l|l|}\hline
\multicolumn{1}{|c|} p & q & r & ($\neg$ p) & ($\neg$ r) & (p $\iff$ q) & ($\neg$ p $\iff$ $\neg$ r) & (p $\iff$ q) $\oplus$ ( $\neg$ p $\iff$ $\neg$ r) \\ \hline
T & T & T & F & F & T & T & F \\ \hline
T & F & T & F & F & F & T & T \\ \hline
T & T & F & F & T & T & F & T \\ \hline
T & F & F & F & T & F & F & F \\ \hline
F & T & T & T & F & F & F & F  \\ \hline
F & T & F & T & T & F & T & T   \\ \hline
F & F & T & T & F & T & F & T    \\ \hline
F & F & F & T & T & T & T & F   \\ \hline
\end{array}
\end{displaymath}
\newpage
\subproblem{c} (p $\oplus$ q) $\Rightarrow$ (p $\oplus$ $\neg$ q)
\solution
\newline
\begin{displaymath}[]
\begin{array}{|l|l|l|l|l|l|}
\hline
\multicolumn{1}{|c|}{p} & q & $\neg$ q & (p $\oplus$ q) & (p $\oplus$ $\neg$ q) & (p $\oplus$ q) $\Rightarrow$ (p $\oplus$ $\neg$ q) \\ \hline
T & T & F & F & T & T\\ \hline
T & F & T & T & F & F \\ \hline
F & T & F & T & F & F \\ \hline
F & F & T & F & T & T  \\ \hline
\end{array}
\end{displaymath}
\newpage
\problem{3: Logic in Algorithms }{10+10+10=30}
If x = 1 before the statement is reached, what is the value of x after each of these statements is encountered in a computer program? Why? Show your work step by step.
\subproblem{a}  \textbf{for} i $\Leftarrow$ 1 \textbf{to} 10 \textbf{do}\\ 
\par \hspace{5mm} \textbf{if} x + 2 = 3 \textbf{then} x := x + 1 \newline
\par \hspace{0.5mm} \textbf{end}
\newline
\solution
%%%%%%REMOVE \newline commands while writing your answer%%%%%
\newline
1)i=1, X=1 is true because of x+2 = 3; then x=2\newline
2)i=2,X=2 is false because of x+2 = 4 ; then x=2\newline
3)i=3,X=2 is false because of x+2 = 4 ; then x=2\newline
.\newline
.\newline
.\newline
9)i=9, x=2 is false because of x+2= 4 ; then x=2\newline
10)i=10,x=2 is false because of x+2=4 ; then x=2\newline
X=2  Because, after step 2, X does not provide the condition given in any step.\newline
\subproblem{b}  \textbf{for} i $\Leftarrow$ 1 \textbf{to} 5 \textbf{do}\\ 
\par \hspace{5mm} \textbf{if} (x + 1 = 2) XOR (x + 2 = 3) \textbf{then} x := x + 1
\newline
\par \hspace{0.5mm} \textbf{end}
\newline
\solution
%%%%%%REMOVE \newline commands while writing your answer%%%%%
\newline
1)i=1 , x=1 , (x+1 = 2) XOR (x+2=3) : 1 XOR 1 =FALSE x=1 still\newline
2)i=2 , x=1 , (x+1 = 2) XOR (x+2=3) : 1 XOR 1 = FALSE x=1 still\newline
3)i=3 , x=1 , (x+1 = 2) XOR (x+2=3) : 1 XOR 1 =FALSE x=1 still\newline
4)i=4 , x=1 , (x+1 = 2) XOR (x+2=3) : 1 XOR 1 =FALSE x=1 still\newline
5)i=5 , x=1 , (x+1 = 2) XOR (x+2=3) : 1 XOR 1 =FALSE x=1 still\newline
X always be 1 because in first step x dont provide the condition in any step
\newline
\subproblem{c}  \textbf{for} i $\Leftarrow$ 1 \textbf{to} 4 \textbf{do} \newline
\par \hspace{5mm} \textbf{if} (2x + 3 = 5) AND (3x + 4 = 7) \textbf{then} x := x + 1
\newline
\par \hspace{0.5mm} \textbf{end}
\newline 
\solution
%%%%%%REMOVE \newline commands while writing your answer%%%%%
\newline
1)i=1, x=1 (2x+3 = 5) AND (3x + 4 = 7) : 1 AND 1 =TRUE , X:=X+1,X=2\newline
2)i=2, x=2 (2x+3 = 5) AND (3x + 4 = 7) : 0 AND 0 =FALSE , X=2 still\newline
3)i=3, x=2 (2x+3 = 5) AND (3x + 4 = 7) : 0 AND 0 =FALSE , X=2 still\newline
4)i=4, x=2 (2x+3 = 5) AND (3x + 4 = 7) : 0 AND 0 =FALSE , X=2 still\newline
\newpage
\problem{4: Proof by contradiction }{20}
Show that at least three of any 25 days chosen must fall in the same month of the year using a proof by contradiction. Explain your work step by step.
\solution
%%%%%%REMOVE \newline commands while writing your answer%%%%%
\newline
1)p:At least three of any 25 days chosen must fall in same mounth of the year\newline
2)$\neg$p:At least three of any 25 days chosen don't have to fall in same mounth of the year\newline
3)If we detirmine $\neg$p is false we prove that p is true\newline
4)1 year = 12 mounth\newline
According to $\neg$p every mounths of the year must has 2 of any 25 days chosen
but if we choose 2 day 12*2 =24 , In this situation, we need to choose one more so $\neg$p is false\newline
5)p is true
\newline
\problem{5: Proof by contraposition }{20}
Show that if $n^3 + 5$ is odd, then n is even using a proof by contraposition. Explain your work step by step.\\
\textit{Note: Assume that n is an integer.}
\solution
%%%%%%REMOVE \newline commands while writing your answer%%%%%
\newline
1)If we prove that ( $\neg$ q $\Rightarrow$ $\neg$ p)  is true we also prove that p $\Rightarrow$ q true\newline\newline
2) $\neg$ q $\Rightarrow$ $\neg$ p :If n is odd then n3+5 is even\newline
3)Assume that n is odd true\newline
n=2k+1  k\varepsilon Z\newline
4)n^3+5=(2k+1)^3 +5=8k^3 + 12k^2 + 6k +6 =2(4k^3 + 6k^2+3k+3)\newline
p=4k^3 + 6k^2+3k+3 ,k   \varepsilon  Z, p\varepsilon Z \newline
5)n^3+5=2p\newline
6)($\neg$ q $\Rightarrow$ $\neg$ p) true\newline
also p $\Rightarrow$ q true
\end{document} 
